\section{Introduction}
% -----------------------------------------------------------
\stoptocentries  % Personal macro: Following sections shouldn't appear in toc
% -----------------------------------------------------------
\subsection*{Contents of the thesis}\;

The Hodge Decomposition theorem for compact Kähler manifolds is a fundamental theorem of the Hodge
Theory. It provides a decomposition of the de Rahm cohomology groups into suitable Dolbeault
cohomology groups, thus yielding a connection between the topology and the complex structure of a
compact Kähler manifold.
\begin{thm}[Hodge Decomposition]
	For a compact Kähler manifold $X$, there is a direct sum decomposition
	\begin{align*}
		H^k_{dR}(X,\mathbb{C}) = \bigoplus_{p+q=k} H^{p,q}_{\opartial}(X,\mathbb{C}).
	\end{align*}
\end{thm}
The primary objective of this thesis will be the elaboration of the proof of this fundamental
theorem. In order to achieve this, we will have to introduce the needed theory first. We are going
to start by presenting the consequences of the existence of an almost complex structure and a
compatible euclidean inner product on a real vector space. For this purpose, we will mainly use the
tools of Linear Algebra.

With this, we will be able to define the local versions of the \emph{Hodge star operator}
$\hodgestar$ and the \emph{Lefschetz} and dual \emph{Lefschetz operators} $L$ and $\Lambda$.

Afterwards, our focus is going to shift to complex manifolds and their different tangent bundles.
Although it is assumed that the reader is already familiar with the definition and basic properties
of complex manifolds, we will begin with the definition and also the elaboration of the properties
of hermitian manifolds, which are the complex counterparts of Riemannian manifolds.

After we have used our local findings for the operators mentioned above to define similarly named
global operators for hermitian manifolds, we will also introduce an \emph{$L^2$-metric} that will be
used to generalize the idea of adjoint operators to \emph{formal adjoint opertors}. We will be
particularly interested in the formal adjoint operators of the exterior derivative $d$ and the
Dolbeault operators $\partial$ and $\opartial$, which will be noted as $d^*,\partial^*$ and
$\opartial^*$. 

Those formal adjoint operators will particularly interest us because they appear in the \emph{Kähler
	identities}. These identities relate the Dolbeault operators and their formal adjoints to each other
using the dual Lefschetz operator. 
\begin{thm}
	On a compact Kähler manifold, we have the identities
	\begin{align*}
		[\Lambda,\opartial] = -i\partial^*, \quad [\Lambda,\partial] = i\opartial^*,
	\end{align*}
	with the Lie bracket being defined as the commutator.
\end{thm}
Next, we are going to introduce the theory of \emph{harmonic differential forms}. In order to do so,
we will define the \emph{Laplacians} $\Delta_d, \Delta_\partial$ and $\Delta_\opartial$ and work out
their properties. We will then use the \emph{Kähler identities} to prove the next important theorem.
\begin{thm}
	For the Laplacians $\Delta_d, \Delta_\partial$ and $\Delta_\opartial$ on a compact Kähler manifold,
	we have the following relation
	\begin{align*}
		\frac{1}{2}\Delta_d = \Delta_\partial = \Delta_\opartial.
	\end{align*}
\end{thm}
Since harmonic and $\Delta_\opartial$-harmonic forms will be defined as forms annihilated by
$\Delta_d$ and $\Delta_\opartial$, respectively, this theorem shows that those two notions are
equivalent for Kähler manifolds. Furthermore, we will use this theorem to establish the following
corollary, which will be crucial for proving the \emph{Hodge Decomposition} theorem.
\begin{cor}
	For the compact Kähler manifold $X$, the complex harmonic differential $k$-forms $\mathcal{H}^k(X)$
	decompose as 
	\begin{align*}
		\mathcal{H}^k(X) = \bigoplus_{p+q=k}\mathcal{H}^{p,q}(X),
	\end{align*}
	with $\mathcal{H}^{p,q}(X)$ being the harmonic differential forms of type $(p,q)$.
\end{cor}
The final statements needed for our proof of the \emph{Hodge Decomposition} theorem will be the
\emph{Hodge Isomorphism theorems}, which enable us to apply the findings of the harmonic forms
theory to the de Rahm and Dolbeault cohomologies by providing two isomorphisms.
\begin{thm}[Hodge Isomorphism theorem \MakeUppercase{\romannumeral 1}]
	The natural mapping
	\begin{align*}
		\mathcal{H}^k(X) \rightarrow H^k_{dR}(X,\mathbb{C}), \;\enspace
		\alpha \mapsto [\alpha]
	\end{align*}
	is an isomorphism. In particular, any class of closed forms in $H^k_{dR}(X,\mathbb{C})$ has a
	unique harmonic
	representative.
\end{thm}

\begin{thm}[Hodge Isomorphism theorem \MakeUppercase{\romannumeral 2}]
	The natural mapping
	\begin{align*}
		\mathcal{H}^{p,q}(X) \rightarrow H^{p,q}_\opartial(X,\mathbb{C}),\; \enspace \alpha \mapsto
		[\alpha]
	\end{align*}
	is an isomorphism. In particular, any class of $\opartial$-closed forms in
	$H^{p,q}_\opartial(X,\mathbb{C})$
	has a unique harmonic representative.
\end{thm}
After these two isomorphism theorems are proven, we already have the \emph{Hodge Decomposition}
given as an isomorphism and in order to get the above \emph{Hodge Decomposition} theorem, we only
need to prove the independence of the Kähler metric.

To conclude this thesis, we will then provide an application of the \emph{Hodge Decomposition}. We
are going to show that the \emph{Hopf surfaces}, which are compact 2-dimensional complex manifolds,
can not be equipped with a Kähler metric.

\subsection*{Remarks on the implementation}\;

For the \emph{Hodge Decomposition}, several different proofs are already known. Therefore, the idea
of this thesis is the collection and the coherent presentation of one of these possible proofs from
the perspective of an undergraduate student who is already familiar with some of the basic concepts
of complex and algebraic geometry.

In this context, this thesis broadly follows the proof presented in \emph{Claire Voisin's} book
\emph{Hodge Theory and Complex Algebraic Geometry I}. However, other popular sources have also
influenced this thesis. Therefore, we will reference similar or equal statements in this literature
whenever possible. This is done to allow for the possibility of verification and to encourage the
reader to engage more deeply with the content.

Also, since multiple different notation conventions exist in complex geometry, we will try to stick
to the notation suggested by \emph{Voisin} in her book. However, we will also provide the reader
with explanations for the used notation throughout the thesis so that even the unfamiliar reader
will be able to follow.

\subsection*{Conventions}\;

Throughout the thesis, we are always going to limit our discussion to differentiable manifolds
without border. In order to have a Riemannian metric on every differentiable manifold, we are also
only going to allow paracompact manifolds. 

Additionally, we are going to adhere to the following meaning of the used symbols.

\begin{table}[ht]
	\centering
	\begin{tabular}{cp{0.6\textwidth}}
		$\mathbb{N}$ & Natural numbers, including 0\\
		$\subset$ & Not necessarily proper subset \\
		$\subsetneq$ & Proper subset\\
		$\into$ & Injection or monomorphism\\
		$\onto$ & Surjection or epimorphism\\
	\end{tabular}
\end{table}

Also, the uppercase letters $J$ and $K$ will usually denote multi-indices except for one instance,
when $J$ denotes an almost complex structure. Additionally, if there are uppercase letters in the
index, they also denote multi-indices. For a basis vector $v_j$, we will write $v^j$ for the dual
basis vector.

\subsection*{Acknowledgments}\;

I want to thank all the nice people for helping me (...)
\vspace*{1.3cm}
\begin{flushright}
	\emph{First name last name}\\
	\emph{Freiburg im Breisgau}\\
	\emph{August 10, 2023}
\end{flushright}

% -----------------------------------------------------------
\starttocentries   % Personal macro: Add following sections to toc
% -----------------------------------------------------------




