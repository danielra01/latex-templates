\section{\for{toc}{German Summary}\except{toc}{Zusammenfassung}}
\begin{otherlanguage}{ngerman}
Diese Arbeit beschäftigt sich mit der Hodge-Zerlegung für kompakte Kähler-Mannigfaltigkeiten,
welche eine der zentralen Aussagen der Hodge-Theorie ist. Sie liefert eine Zerlegung der
de-Rahm-Kohomologie-Gruppen in passende Dolbeault-Kohomologie-Gruppen und stellt somit eine
Verbindung zwischen den topologischen Eigenschaften und der komplexen Struktur einer kompakten
Kähler-Mannigfaltigkeit her.

Das Ziel dieser Arbeit ist die Ausarbeitung des Beweises dieser Zerlegung. Dafür muss die
erforderlich Theorie eingeführt und erklärt werden. Dabei ist es zunächst sinnvoll, die lokale
Theorie auszuarbeiten. Diese befasst sich hauptsächlich mit den Eigenschaften von euklidischen und
unitären Vektorräumen im Zusammenhang mit der Existenz einer kompatiblen fastkomplexen Struktur.
Dabei werden vor allem die Werkzeuge aus der Linearen Algebra gebraucht.

Mit den gesammelten Eigenschaften werden dann jeweils der Lefschetz-Operator, der duale
Lefschetz-Operator und der Hodge-Stern-Operator lokal definiert.

Danach wird der Fokus zunehmend auf Mannigfaltigkeiten gelegt. Nachdem hermitesche
Mannigfaltigkeiten definiert wurden, werden einige der zuvor erarbeiteten lokalen Aussagen in
globale Aussagen übersetzt. Außerdem werden die entsprechenden globalen Operatoren definiert. Dabei
wird jedoch angenommen, dass der Leser bereits mit den grundlegenden Begriffen und Eigenschaften von
komplexen und fastkomplexen Mannigfaltigkeiten vertraut ist.

Nachdem formal adjungierte Operatoren mithilfe einer vorher definierten $L^2$-Metrik eingeführt
wurden, werden dann die Kähler-Identitäten behandelt. Diese stellen die zuvor eingeführten globalen
Operatoren in Relation zueinander und sind äußerst essenzielle Eigenschaften von
Kähler-Mannigfaltigkeiten. Diese Kähler-Identitäten werden in dieser Arbeit jedoch nicht bewiesen.

Das nächste Ziel ist der Beweis der Hodge-Isomorphie-Sätze. Dafür wird die Theorie der harmonischen
Differentialformen eingeführt und einige wichtige Eigenschaften werden bewiesen. Dafür werden die
zuvor behandelten Kähler-Identitäten benötigt.

Danach wird mithilfe dieser Isomorphie-Sätze die Hodge-Zerlegung bewiesen. Außerdem wird gezeigt,
dass diese Zerlegung unabhängig von der Wahl der Kähler-Metrik ist. Am Ende wird dann eine nützliche
topologische Anwendung der Hodge-Zerlegung präsentiert.
\end{otherlanguage}
