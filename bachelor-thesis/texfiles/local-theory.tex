\section{Local Theory}
This chapter aims to provide the reader with some of the essential notions and tools for the local
theory needed to form a comprehensive understanding of the concept of hermitian manifolds and 
in particular, Kähler manifolds. The main focus will be on complex vector spaces and hermitian 
forms on those spaces and also the tools linear algebra provides.

Later, we are going to focus on the different tangent bundles of complex manifolds, which are
collections of vector spaces that vary in a geometric way on the manifold. Therefore, most of the 
notions and tools we are going to introduce in this chapter will be translated into the global 
context later.

The main goal of this chapter will be the definition of the local Hodge star operator \nolinebreak
$\hodgestar$ and the definition of the local Lefschetz and dual Lefschetz operator $L$ and 
$\Lambda$. For this purpose, we will start by providing a brief overview of the topic of 
complexification of vector spaces. Afterwards, we will focus on euclidean and hermitian vector 
spaces. Finally, we will conclude with the definition of the operators mentioned above.

The reader should be aware that this chapter is based on the similarly named chapter in Daniel
Huybrechts' \emph{Complex Geometry} \cite{Huybrechts2004}, although the level of detail found there
is not to be expected.

\subsection{Complexification of vector spaces}\;

To simplify the notation, we are going to assume the following setting for the remainder of this
section.
\begin{set}
	Let $V$ denote a real $n$-dimensional vector space. Also, assume that $V$ is an almost complex
	vector space, i.e. $V$ is equipped with an 
	endomorphism $I: V\rightarrow V$ such that $I^2 = -\id$. This endomorphism is called the almost
	complex structure of $V$.
\end{set}

Using this almost complex structure, we can also think of $V$ as a complex vector space with the
$\mathbb{C}$-module structure defined as $(a+ib)\cdot v = a v + b I(v)$ for all $a,b \in
\mathbb{R}$. For this complex vector space, we will write $(V,I)$. With the product rule for the
determinant, we can calculate
\begin{align*}
	\det(I)^2 = \det(I^2) = \det(-\id) = (-1)^n.
\end{align*}
Since $\det(I)$ is real, we conclude that $n = 2m$ for some $m \in \mathbb{N}$.

Furthermore, $V$ and $(V,I)$ are equal as sets and if $(v_1,\dots,v_d)$ is a complex basis of
$(V,I)$, it is immediate that $(v_1,I(v_1),\dots,v_d,I(v_d))$ is a real basis of $V$. Therefore,
their dimensions relate as
\begin{align*}
	\dim_\mathbb{C} (V,I) = d = \frac{1}{2} \dim_\mathbb{R} V = \frac{1}{2}n = m.
\end{align*}

Additionally, as an almost complex vector space, $V$ is endowed with a natural orientation. This
boils down to the fact that the real space $\mathbb{C}^m$ has a natural orientation given by the
basis $(e_1, ie_1, \dots e_m, ie_m)$, with the $e_1,\dots,e_m$ being the standard basis vectors (cf.
\cite[Corollary 1.2.3]{Huybrechts2004}).

At the same time, it is possible to construct a different complex vector space using $V$. 
\begin{defn}
	The \emph{complexification} $V_\mathbb{C}$ of $V$ is defined as
	$V_\mathbb{C} := V \otimes_\mathbb{R} \mathbb{C}$.
\end{defn}
Let $\left(v_1, \dots, v_n\right)$ be a real basis of $V$. With the properties of the tensor
product, it is $\left(v_1 \otimes 1, \dots , v_n \otimes 1\right)$ a complex basis of
$V_\mathbb{C}$. This shows that there exists an inclusion $V \into V_\mathbb{C}$ and for the
dimension of $V_\mathbb{C}$, we get
\begin{align*}
	\dim_\mathbb{C} V_\mathbb{C} = n = \dim_\mathbb{R} V.
\end{align*}
Now, the almost complex structure $I$ can be linearly extended to an almost complex structure
$I_\mathbb{C}$ on $V_\mathbb{C}$. This is defined as $I_\mathbb{C} (v\otimes 1 + w \otimes i) :=
I(v) \otimes 1 + I(w)\otimes i$, and it is evident that this linear extension also has the property
$I_\mathbb{C}^2 = -\id$. Thus, we also call this to be an almost complex structure.
\begin{nota}
	Note that for a vector $v\otimes \lambda \in V_\mathbb{C}$, it is a common practice to sometimes
	omit the tensor product in the notation, just noting $\lambda v$ instead of $v \otimes \lambda$. If
	it is possible without confusion, we will also write $I$ instead of $I_\mathbb{C}$ for the complex
	extension of the almost complex structure.
\end{nota}
The following proposition shows how the two $\mathbb{C}$-module structures on $V_\mathbb{C}$,
defined by the almost complex structure $I$ and by multiplication with $i$, compare to each other.
\begin{prop}[Decomposition of $V_\mathbb{C}$ {\cite[Lemma 1.2.5]{Huybrechts2004}}]
	\label{loc-theory:lm:decomposition-of-vc}
	For the complexification $V_\mathbb{C}$ we have the decomposition 
	$V_\mathbb{C} = V^{1,0} \oplus V^{0,1}$ with 
	\begin{align*}
		V^{1,0} := \left\{v \in V_\mathbb{C} \mid I(v) = iv\right\} \enspace
		\text{and} \enspace V^{0,1} := \left\{v \in V_\mathbb{C} \mid I(v) =- iv\right\}.
	\end{align*}
\end{prop}
\begin{proof}We extend the proof of the stated lemma in \cite{Huybrechts2004}.
	Let $v \in V_\mathbb{C}$. It is $v = \frac{1}{2} (v- iI(v)) + \frac{1}{2} (v + iI(v))$. A simple
	calculation shows
	\begin{align*}
		I(v-iI(v)) = I(v) -I(iI(v)) = I(v) - iI^2(v) = I(v) + iv = i(-iI(v) + v)
	\end{align*}
	and thus $\frac{1}{2} (v -iI(v)) \in V^{1,0}$. With a similar calculation we obtain 
	$\frac{1}{2} (v+iI(v)) \in V^{0,1}$. At the same time, it holds to be $V^{1,0} \cap V^{0,1} = \{0\}$. 
	Thus, the inclusion $V^{1,0} \oplus V^{0,1} \into V_\mathbb{C}$ is injective and with the above
	calculation, it is also surjective. Hence, it is a canonical isomorphism, so the decomposition is
	proven.
\end{proof}
\begin{rem} We expand the argument in the proof of \cite[Lemma 1.2.5]{Huybrechts2004}.
	The proof of the last proposition shows that a vector $w \in V^{1,0}$ can be written as 
	$w = v-iI(v)$ for some $v \in V_\mathbb{C}$. At the same time, we can split $v = x + iy$ 
	with $x,y \in V$. Then it is 
	\begin{align*}
		\overline{w} = \overline{v -iI_\mathbb{C}(v)} &= \overline{x + iy - i (I(x) + iI(y))}\\
		&= x -iy+iI(x)+I(y) = \overline{v}+ i(I(x) + -iI(y)) = \overline{v} + iI_\mathbb{C}(\overline{v}).
	\end{align*}
	Hence it is $\overline{w} \in V^{0,1}$. Similar calculations show that for $w \in V^{0,1}$, it is
	$\overline{w} \in V^{1,0}$ and $\overline{\widebar{w}} = w$. Since complex conjugation is
	$\mathbb{R}$-linear, this already proves that $V^{1,0}$ and $V^{0,1}$ are isomorphic as real vector
	spaces.
\end{rem}
\begin{rem}
	\label{loc-theory:rem:c-linear-c-antilinear}
	Using the proof of the last proposition and the natural inclusion \linebreak$V \into V_\mathbb{C},\,
	v \mapsto v \otimes 1$, we can define an $\mathbb{R}$-linear isomorphism 
	\begin{align*}
		\varphi_1: (V,I) \rightarrow V^{1,0},\;\enspace v \mapsto \big(v\otimes 1-iI_\mathbb{C}(v\otimes 1)\big).
	\end{align*}
	However, we are able to calculate
	\begin{align*}
		\varphi_1(I(v)) = I(v) \otimes 1 - iI_\mathbb{C}(I(v) \otimes 1) = I_\mathbb{C}(v \otimes 1)
		-iI_\mathbb{C}^2(v \otimes 1) = I_\mathbb{C}\big(v\otimes 1 -iI_\mathbb{C}(v\otimes 1)\big).
	\end{align*}
	Hence, we obtain $\varphi_1(I(v)) = I_\mathbb{C}(\varphi_1(v)) = i\varphi_1(v)$ because
	$\varphi_1(v) \in V^{1,0}$. Since the $\mathbb{C}$-module structure on $(V,I)$ is defined using $I$,
	we know that $\varphi_1$ is also a $\mathbb{C}$-linear isomorphism. At the same time, we are able to
	define a similar $\mathbb{R}$-linear isomorphism 
	\begin{align*}
		\varphi_2: (V,I) \rightarrow V^{0,1} \;\enspace
		v \mapsto (v \otimes 1 + iI_\mathbb{C}(v \otimes 1)).
	\end{align*}
	The same calculation yields $\varphi_2(I(v)) = -i\varphi_2(v)$. Thus, $\varphi_2$ is a
	$\mathbb{C}$-antilinear isomorphism.
\end{rem}
Next, we are going to define an induced almost complex structure on the dual space $V^*$. Because of
its induced nature, this almost complex structure is also called $I$ and it is defined as a mapping 
$I: V^* \rightarrow V^*$, such that $I(f)(v) = f(I(v))$ for all $f \in V^*$ and $v \in V$.
Now, the following lemma ensures the compatibility of the complexification with the dual space of $V$.
\begin{lm}[{\cite[Lemma 1.2.6]{Huybrechts2004}}]
	\label{loc-theory:lm:compatibility-of-dual-and-complexification}
	It is $(V_\mathbb{C})^* = \Hom_\mathbb{R}(V,\mathbb{C}) = (V^*)_\mathbb{C}$ and it also holds to be
	\begin{alignat*}{2}
		(V^{1,0})^* &= \left\{f \in \Hom_\mathbb{R}(V,\mathbb{C}) \mid f(I(v)) = if(v)\; \forall v \in V\right\} = (V^*)^{1,0},\\
		(V^{0,1})^* &= \left\{f \in \Hom_\mathbb{R}(V,\mathbb{C}) \mid f(I(v)) = -if(v)\;  \forall v \in V\right\} = (V^*)^{0,1}.
	\end{alignat*}
\end{lm}
\begin{proof}
	It is $(V_\mathbb{C})^* = \Hom_\mathbb{C} (V_\mathbb{C}, \mathbb{C})$ and 
	$(V^*)_\mathbb{C} = \Hom_\mathbb{R} (V, \mathbb{C})$. In order to prove that these two spaces 
	are equal, we have to prove the existence of a canonical isomorphism 
	$\Hom_\mathbb{R}(V,\mathbb{C}) \cong \Hom_\mathbb{C} (V_\mathbb{C}, \mathbb{C})$.
	
	Let $f \in \Hom_\mathbb{R}(V,\mathbb{C})$ and extend it to an $\mathbb{R}$-linear mapping
	$\tilde f: V_\mathbb{C} \rightarrow \mathbb{C}$ by setting $\tilde f(v \otimes \lambda): = \lambda
	f(v)$ for all $v \in V$ and $\lambda \in \mathbb{C}$. This mapping is also $\mathbb{C}$-linear
	because we can show for all $\mu \in \mathbb{C}$
	\begin{align*}
		\tilde f(\mu\cdot(v \otimes \lambda)) = \tilde f (v \otimes \mu\lambda) = \mu\lambda f(v) = \mu
		\cdot \tilde f (v \otimes \lambda).
	\end{align*}
	This shows that for every $f$, we can find a unique $\tilde f \in \Hom_\mathbb{C}(V_\mathbb{C},
	\mathbb{C})$.\\
	Let now $g \in \Hom_\mathbb{C}(V_\mathbb{C}, \mathbb{C})$. Using the inclusion 
	$V \into V_\mathbb{C}$, we can restrict $g$ to obtain a mapping $h: V \rightarrow \mathbb{C}$ 
	that is defined as $h(v):= g(v \otimes 1)$.
	Since $g$ was $\mathbb{C}$-linear, $h$ is already an $\mathbb{R}$-linear mapping.
	
	This shows that $h \in \Hom_\mathbb{R}(V,\mathbb{C})$, and since those two constructions are
	obviously inverse to each other, this completes the proof of the first statement.\footnote{This
		first part of the proof was created using a Math Stack Exchange post of the user \emph{Mark}
		(\url{https://math.stackexchange.com/users/470733/mark}) that can be found on
		\url{https://math.stackexchange.com/q/4718935} and was last checked on the 25th of August, 2023.}\\
	For the second statement, we use \Cref{loc-theory:rem:c-linear-c-antilinear} to get
	\begin{align*}
		(V^{1,0})^* = \Hom_\mathbb{C}(V^{1,0},\mathbb{C})&= \Hom_\mathbb{C}((V,I),\mathbb{C}) \\&=
		\left\{f \in \Hom_\mathbb{R}(V,\mathbb{C}) \mid f(I(v)) = if(v)\; \forall v \in V\right\}.
	\end{align*}
	Additionally, for the other subspace, we can use the same remark to obtain
	\begin{align*}
		(V^{0,1})^* = \Hom_\mathbb{C}(V^{0,1},\mathbb{C}) &= \Hom_{\overline{\mathbb{C}}}((V,I),
		\mathbb{C}) \\&= \left\{f \in \Hom_\mathbb{R}(V,\mathbb{C})\mid f(I(v)) = -if(v)\; \forall v \in
		V\right\}.
	\end{align*}
\end{proof}
\begin{nota}
	Because of the last lemma, we will only write $V^*_\mathbb{C}$, omitting the brackets from now on.
\end{nota}
\subsection{Euclidian and hermitian vector spaces}\;

Later, we are going to define the notion of a hermitian manifold, i.e. a complex manifold whose
holomorphic tangent space in every point is equipped with a hermitian form. In order to do so, this
section will cover some fundamental statements about those forms on complex vector spaces.

For the remainder of this section, we are going to assume the following setting.
\begin{set}
	Let $(V,g)$ be a real $n$-dimensional euclidean vector space, i.e. $g$ is a positive definite
	symmetric bilinear form on the real space $V$. Also, assume that $V$ is equipped with an almost
	complex structure $I$.
\end{set}
\begin{defn}
	The inner product $g$ is said to be \emph{compatible with the almost complex structure} $I$ if it
	holds to be $g(I(v),I(w)) = g(v,w)$ for all $v,w \in V$.
\end{defn}
\begin{nota}
	If the inner product $g$ on $V$ is compatible with the almost complex structure $I$, we usually
	only write $(V,g,I)$.
\end{nota}
The just-established notion of a compatible inner product gives rise to an additional notion.
\begin{defn}
	The \emph{fundamental form} associated to $(V,g,I)$ is defined as the form $\omega \in
	\bigwedge^2V^* \cap \bigwedge^{1,1} V^*$, such that for all $v,w \in V$ it is
	\begin{align*}
		\omega(v,w) := g(I(v),w) = - g(v, I(w)).
	\end{align*}
	Note that the second equality is equivalent to $g$ being compatible with the almost complex
	structure $I$. Also, note that this immediately yields $\omega(I(v),I(w)) = \omega(v,w)$.
\end{defn}
\begin{rem}
	The expression $ \bigwedge^2V^* \cap \bigwedge^{1,1} V^*$ has to be explained. With
	\Cref{loc-theory:lm:compatibility-of-dual-and-complexification}, we know that 
	$\bigwedge^2 V^*\subset \bigwedge^2 V_\mathbb{C}^*$. At the same time, it is
	\begin{align}
		\label{loc-theory:eq:decomps-of-complex-2-forms}
		\bigwedge\nolimits^2 V_\mathbb{C}^* = \bigwedge\nolimits^{2,0} V^* \oplus \bigwedge\nolimits^{1,1}
		V^* \oplus \bigwedge\nolimits^{0,2} V^*
	\end{align}
	(cf. \cite[Proposition 1.2.8 (ii), Example 1.2.34]{Huybrechts2004}).
	and for these reasons, the intersection is meaningful as it happens in $\bigwedge^2V^*_\mathbb{C}$.
	This expression describes all the alternating real 2-forms of type $(1,1)$, i.e. alternating real
	2-forms on $V$, that are also $\mathbb{C}$-linear in it's first argument and $\mathbb{C}$-antilinear
	in its second argument if viewed as forms on $V_\mathbb{C}$.
\end{rem}
Now, for the fundamental form $\omega$ from the last definition to be well-defined, we have
to check whether $\omega$ is indeed an alternating real 2-form on $V$ and is also of type $(1,1)$.
For the first statement, real bilinearity follows directly with the bilinearity of $g$. Also using the symmetry
of $g$, we calculate for all $v,w \in V$
\begin{align*}
	\omega(v,w) = g(I(v),w) = g(I^2(v),I(w)) = - g(v,I(w)) = - g(I(w),v) = -\omega(w,v).
\end{align*}
Hence, $\omega$ is alternating and therefore a real 2-form. With
\Cref{loc-theory:eq:decomps-of-complex-2-forms}, it suffices to show that the $\mathbb{C}$-bilinear
extension of $\omega$ vanishes on all pairs of vectors $v,w$ in $V^{1,0}$ or $V^{0,1}$ to
prove that it is of type $(1,1)$. In the first case, i.e. $v,w \in V^{1,0}$, we calculate
\begin{align*}
	\omega(v,w) = \omega(I(v),I(w)) = \omega(iv,iw) = i^2 \omega(v,w)  = -\omega(v,w).
\end{align*}
Hence $\omega(v,w) = 0$. The first equation holds because the complex bilinear extension
inherits this property from the real form $\omega$. The calculation for the other case can be carried out 
analogously. Hence, $\omega$ is indeed of type $(1,1)$ and this establishes the well-definedness of 
$\omega$.

For $(V,g,I)$, we can also define a positive definite hermitian form on the complex space $(V,I)$. 
This form is defined as 
\begin{align*}
	h: (V,I) \times (V,I) \rightarrow \mathbb{C},\;\enspace
	(v,w) \mapsto g(v,w) - i\omega(v,w).
\end{align*}
Additionally, the inner product $g$ on $V$ can be extended sesquilinearly to a positive definite
hermitian form on $V_\mathbb{C}$. This extension is defined as
\begin{align*}
	h_\mathbb{C} : V_\mathbb{C} \times V_\mathbb{C} \rightarrow \mathbb{C},\; \enspace 
	(v \otimes \lambda, w \otimes \mu) \mapsto (\lambda \overline{\mu}) \cdot g(v,w).
\end{align*}
See also \cite[p.\,30]{Huybrechts2004} for the similar definitions.
However, it has to be checked whether these two positive definite hermitian forms are well-defined.
\begin{prop}[{\cite[Lemma 1.2.15]{Huybrechts2004}}]
	For $(V,g,I)$, the form $h:= g- i\omega$ is indeed a positive definite
	hermitian form on $(V,I)$. Also, the extension $h_\mathbb{C}$ of the inner product $g$ defines a
	positive definite hermitian form on $V_\mathbb{C}$.
\end{prop}
\begin{proof}
	We expand the calculation in the proof of the stated lemma in \cite{Huybrechts2004} and add the
	argument for our second statement. Let 
	$v,w \in (V,I)$. It is $h(v,v) = g(v,v) - i \omega(v,v) =g(v,v)$ because $\omega$ is alternating and 
	therefore $\omega(v,v) = 0$. Since $g$ is positive definite, this proves that $h$ is positive definite 
	as well. Furthermore, since $g$ is symmetric, it is also
	\begin{align*}
		h(v,w) = g(v,w) - i\omega(v,w) = g(w,v) + i\omega(w,v) = \overline{h(w,v)},
	\end{align*}
	and it also holds to be
	\begin{align*}
		h(I(v),w) &= g(I(v),w) - i\omega(I(v),w) \\&= g(I^2(v),I(w)) -i (g(I^2(v),w)) \\&= -g(v,I(w)) +
		ig(v,w) \\&= i\big(i g(v,I(w)) + g(v,w)\big) \\&= i\big(- i\omega(v,w) + g(v,w)\big) = i h(v,w).
	\end{align*}
	On $(V,I)$, the image under $I$ corresponds to multiplication with $i$ because the
	$\mathbb{C}$-module structure is defined using the almost complex structure. This proves the 
	$\mathbb{C}$-linearity in the first	argument, as the $\mathbb{R}$-linearity is already inherited from 
	$g$ and $\omega$.
	
	For the $\mathbb{C}$-antilinearity in the second argument, we combine the last two statements to
	get
	\begin{align*}
		h(v,I(w)) = \overline{h(I(w),v)} = \overline{ih(w,v)} = -ih(v,w).
	\end{align*} This completes the proof of the first statement. 
	
	To prove the second statement, it is already clear by definition that $h_\mathbb{C}$ is 
	$\mathbb{C}$-linear in its first argument and $\mathbb{C}$-antilinear in its second argument. Let 
	$(v_1,\dots,v_n)$ be an orthonormal basis of $V$ with respect to the inner product $g$. With the 
	properties of the tensor product, it is again $(v_1 \otimes 1, \dots, v_n \otimes 1)$ a basis of 
	$V_\mathbb{C}$. Therefore, we can write every element $u \in V_\mathbb{C}$ as
	$u = \sum_{j=1}^{n} \lambda_j(v_j \otimes 1)= \sum_{j=1}^{n} v_j \otimes \lambda_j$. We are then
	able to calculate
	\begin{align*}
		h_\mathbb{C}(u,u) = h_\mathbb{C}\Big(\sum_{j=1}^n v_j \otimes \lambda_j, \sum_{k=1}^n v_k \otimes
		\lambda_k\Big) = \sum_{j,k=1}^{n} \lambda_j\overline\lambda_k g(v_j,v_k) = \sum_{j=1}^n|\lambda_j|^2
		\geq 0,
	\end{align*}
	
	and $h_\mathbb{C}(u,u) = 0$ if and only if $u = 0$. Hence, $h_\mathbb{C}$ is positive definite.
	Furthermore, it holds to be
	\begin{align*}
		h_\mathbb{C}(v \otimes \lambda,w \otimes \mu) = \lambda \overline{\mu} \cdot  g(v,w) = \overline
		{\overline \lambda \mu \cdot g(v,w)} = 	\overline{\overline \lambda \mu \cdot g(w,v)} =
		\overline{h_\mathbb{C} (w \otimes \mu, v \otimes \lambda)}.
	\end{align*}
	Thus, $h$ and $h_\mathbb{C}$ are both positive definite hermitian forms.
\end{proof}

\begin{nota} In \cite[Lemma 1.2.17]{Huybrechts2004}, it is shown that these two hermitian forms only
	differ by a factor of $\frac{1}{2}$ under the natural inclusion $(V,I) \into V^{1,0}$. This may be a
	reason for the common practice not to differentiate between $h$ and $h_\mathbb{C}$ in the notation,
	which we will also adhere to.
\end{nota}
\begin{rem}
	\label{loc-theory:rem:real-of-hermitian-form}
	In the last proposition, it has been proven that the compatible inner product $g$ already defines a
	positive definite hermitian form on $(V,I$). Let now $\tilde h$ be an arbitrary positive definite hermitian 
	form on $(V,I)$. We can define the \emph{real part} of $h$ as follows.
	\begin{align*}
		\Re(\tilde h)(v,w) := \frac{1}{2} \big(\tilde h(v,w) + \overline{\tilde h(v,w)}\big) =
		\frac{1}{2}\big(\tilde h(v,w) + \tilde h(w,v)\big)
	\end{align*}
	With the second equality, it is obvious that this defines a real positive definite and symmetric
	bilinear form on $V$. It is also
	\begin{align*}
		\Re(\tilde h)(I(v),I(w)) &= \frac{1}{2} \big(\tilde h(I(v),I(w)) + \tilde h(I(w),I(v))\big)\\&=
		\frac{1}{2} \big(\tilde h(v,w) + \tilde h(w,v)\big) \\&= \Re(\tilde h)(v,w)
	\end{align*}
	and thus $\Re(\tilde h)$ defines a compatible inner product on $V$. Because these constructions are
	inverse to each other, there is a one-to-one relation between positive definite hermitian forms on
	$(V,I)$ and inner products on $V$ that are compatible with $I$. See also 
	\cite[Section. 3.1.1]{Voisin2002} for further information about this relation.
\end{rem}

Later, we will also need inner products and hermitian forms on the exterior algebra spaces
$\bigwedge^kV^*$ and $\bigwedge^k V^*_\mathbb{C}$. In the following two lemmas, we are going 
to construct inner products on $V^*$ and  $\bigwedge^kV$ using the existing inner product $g$ on $V$, 
and we will eventually use those constructions to define an induced inner product on
$\bigwedge\nolimits^kV^*$. This construction is inspired by \cite[Section 11]{Schnell2012}, but no
proofs are provided there.

Before we begin, we need to take a look at the natural linear mapping
\begin{align*}
	g^\flat: V \rightarrow V^*, \;v\mapsto g(v,-).
\end{align*}
With respect to the inner product $g$, we can choose an orthonormal basis $(v_1,\dots,v_n)$ of $V$.
For all $r,s \in \mathbb{N}$ with $1 \leq r,s \leq n$, we get
\begin{align*}
	g^\flat(v_r)(v_s) = g(v_r,v_s) = \delta_{rs}.
\end{align*}
Hence, $g^\flat(v_r) = v^r$ with $v^r$ being the dual basis vector of $v_r$. This already shows that
$g^\flat$ is an isomorphism. Let now $g^\sharp: V^* \rightarrow V$ denote the inverse mapping of
$g^\flat$. Using this mapping, we get the following lemma.
\begin{lm}
	\label{loc-theory:lm:product-on-dual-space}
	The inner product $g$ on $V$ induces an inner product on $V^*$. It is defined \nolinebreak as 
	\begin{align*}
		\tilde{g}: V^* \times V^* \rightarrow \mathbb{R},\; \enspace
		(v,w) \mapsto g\big(g^\sharp(v),g^\sharp(w)\big).
	\end{align*}
\end{lm}
\begin{proof}
	It is obvious that $\tilde{g}$ defines a bilinear mapping. Let now $(v_1,\dots,v_n)$ be an
	orthonormal basis of $V$ again. Also let $(v^1,\dots,v^n)$ denote the corresponding dual basis of
	$V^*$. If only evaluated on those dual basis vectors, $\tilde g$ simplifies as
	\begin{align*}
		\tilde{g}(v^r,v^s) = g(g^\sharp(v^r), g^\sharp(v^s)) = g(v_r,v_s).
	\end{align*}
	Thus, $\tilde{g}$ directly inherits the inner product properties from \nolinebreak$g$.
\end{proof}
\begin{lm}
	\label{loc-theory:lm:product-on-exterior-algebra}
	The inner product $g$ on $V$ induces an inner product on $\bigwedge\nolimits^kV$, which is defined as
	\begin{align*}
		g_k: \bigwedge\nolimits^kV \times \bigwedge\nolimits^kV &\rightarrow \mathbb R\\
		\big(v_1 \wedge \dots \wedge v_k, w_1\wedge\dots \wedge w_k\big) &\mapsto
		\det\Big(\big(g(v_r,w_s)\big)_{rs}\Big).
	\end{align*}
\end{lm}
\begin{proof}
	With the multilinearity of the determinant, it is again obvious that $g_k$ is a bilinear mapping.
	Since the determinant is invariant under transposition, $g_k$ is also symmetric. Let now 
	$(v_1,\dots,v_n)$ be an orthonormal basis of $V$ again. We know that this induces a basis 
	$(v_J)_J$ of $\bigwedge^k V$. (cf. \cite[Proposition 14.8]{Lee2012}.) If we only evaluate $g_k$ 
	on these basis vectors again, we get
	\begin{align*}
		g_k(v_J,v_K) = \det\Big(\big(g(v_{j_r}, v_{k_s})\big)_{rs}\Big) =
		\delta_{JK}.
	\end{align*}
	This is because if $J \neq K$, there is at least one $j_l \in J$ such that $j_l \not\in K$. Thus we
	obtain $g(v_{j_l}, v_{\tilde{k}}) = 0$ for all $\tilde{k} \in K$. Hence, the $l$-th column of the 
	matrix $G:=\big(g(v_{j_r},v_{k_s})\big)_{rs}$ is everywhere zero and therefore 
	$\det(G) = 0$.  If $J$ and $K$ are equal however, then it is $G = \id_k$ and therefore 
	$\det(G) = 1$. With this, we have for all $\alpha := \sum_{J} \alpha_J v_J\in \bigwedge^k V$
	\begin{align*}
		g_k(\alpha,\alpha) = \sum_{J}  \alpha_J^2\, g_k(v_J,v_J) = \sum_{J} \alpha_J^2  \geq 0.
	\end{align*}
	This calculation also implies that $g_k(\alpha,\alpha) = 0$ if and only if $\alpha = 0$. Thus, the
	statement is proven.
\end{proof}
The combination of the last two lemmas finally gives us an inner product on
$\bigwedge\nolimits^kV^*$.
\begin{cor}
	\label{loc-theory:cor:induced-product-on-exterior-algebra}
	The induced inner product on $\bigwedge^kV^*$ is given as
	\begin{align*}
		\tilde{g}_k: \bigwedge^kV^* \times \bigwedge^kV^* &\rightarrow \mathbb{R}\\
		\big(v^1 \wedge \dots \wedge v^k, w^1\wedge\dots \wedge w^k\big) &\mapsto
		\det\Big(\big(g(g^\sharp(v^j),g^\sharp(w^k))\big)_{jk}\Big).
	\end{align*}
\end{cor}
\begin{rem}
	\label{loc-theory:rem:preserving-orthonormality}
	The two proofs of
	\Cref{loc-theory:lm:product-on-dual-space,loc-theory:lm:product-on-exterior-algebra}
	also show that the induced inner products preserve orthonormality, i.e. for an orthonormal basis 
	$(v_1,\dots,v_n)$ of $V$, the induced bases $(v^1,\dots,v^n)$ of $V^*$ and $(v_J)_{J}$ of 
	$\bigwedge^k V$ are also orthonormal with respect to the induced inner products. Since \Cref{loc-theory:cor:induced-product-on-exterior-algebra} just combines these inner products, 
	this property also holds for the induced inner product $\tilde{g}_k$ on $\bigwedge\nolimits^kV^*$.
\end{rem}

\begin{rem}
	\label{loc-theory:rem:hermitian-form-on-exterior-algebra}
	Similar to the construction of the positive definite hermitian form on $V_\mathbb{C}$, we are able
	to obtain positive definite hermitian forms $\tilde{h}_k$ on the exterior algebra spaces
	$\bigwedge\nolimits^kV_\mathbb{C}^*$, by extending the inner products $\tilde{g}_k$ sesquilinearly
	(cf. \cite[p.\,33]{Huybrechts2004}).
\end{rem}

\begin{nota}
	It is common practice to only write $g$ and $h$ for all these inner products and hermitian forms,
	respectively. This is because they have similar properties and since they are all defined on different 
	spaces, it should always be clear from the context which form is meant.
\end{nota}

\subsection{Local operators}\;

The next section will focus on the definition of some essential operators. These are initially
defined as local operators on vector spaces but are going to be used on vector bundles later.
However, most of their properties can already be shown locally.

Therefore, we will assume the following setting for the remainder of this section.
\begin{set}
	Let $(V,g, I)$ be an euclidean vector space of dimension $n$ with a compatible almost complex
	structure. Also, let $\omega$ denote the fundamental form associated with $g$ and let $h$ denote
	the induced hermitian forms on $(V,I)$ and $V_\mathbb{C}$. The existence of the almost complex
	structure $I$ ensures $n = 2m$ fore some $m \in \mathbb{N}$.
\end{set}
\begin{defn}[Lefschetz operator]
	With $\omega$ the associated fundamental form to $(V,g,I)$, the real \emph{Lefschetz operator} is
	defined as the linear mapping
	\begin{align*}
		L: \bigwedge\nolimits^k V^* \rightarrow \bigwedge\nolimits^{k+2} V^*,\;\enspace
		\alpha \mapsto \omega \wedge \alpha.
	\end{align*}
	The complex \emph{Lefschetz operator} on $\bigwedge\nolimits^kV^*_\mathbb{C}$ is then
	defined as the $\mathbb{C}$-linear extension $L_\mathbb{C}: \bigwedge\nolimits^k V^*_\mathbb{C}
	\rightarrow \bigwedge\nolimits^{k+2} V^*_\mathbb{C}$, i.e. $L_\mathbb{C}(\beta) = \omega \wedge
	\beta$ for all $\beta \in \bigwedge\nolimits^{k+2} V^*_\mathbb{C}$.
\end{defn}
\begin{rem}
	Note that $\omega \in \bigwedge\nolimits^2 V^* \cap \bigwedge\nolimits^{1,1}
	V^*\subset\bigwedge\nolimits^2V_\mathbb{C}^*$ and this is the reason for these two wedge products 
	to be both meaningful. Also, we have to keep in mind that the definition of this operator depends on the 
	fundamental form $\omega$, which itself depends on the choice of the inner product $g$.
	
	Furthermore, due to  the fundamental form $\omega$ being of type $(1,1)$, it is apparent that the
	restriction of $L_\mathbb{C}$ to forms of type $(p,q)$ behaves like 
	$L: \bigwedge\nolimits^{p,q} V^* \rightarrow \bigwedge\nolimits^{p+1,q+1}V^*$
	and thus the complex Lefschetz operator preserves the revised structure of 
	$\bigwedge\nolimits^k V_\mathbb{C}^* = \bigoplus_{p+q=k}\bigwedge\nolimits^{p,q}V^*$ 
	(cf. \cite[Proposition 1.2.8 (ii)]{Huybrechts2004}).
\end{rem}

In linear algebra, we have the notion of an adjoint operator with respect to an inner product on a
vector space. Therefore, we can use our induced inner product on the exterior algebra spaces
$\bigwedge\nolimits^kV^*$ to define the dual of the Lefschetz operator.
\begin{defn}
	The \emph{dual Lefschetz operator} $\Lambda$ is defined as the adjoint operator of $L$ with respect
	to the inner product $g$, i.e. the uniquely defined mapping
	\begin{align*}
		\Lambda: \bigwedge\nolimits^{k+2} V^* \rightarrow \bigwedge\nolimits^{k} V^*,
	\end{align*}
	such that for all $\alpha \in \bigwedge\nolimits^{k+2} V^*$ and $\beta \in \bigwedge\nolimits^{k}V^*$,
	it is $g(\Lambda(\alpha), \beta) = g(\alpha,L(\beta))$.
\end{defn}
\begin{rem}
	Note that this operator is indeed uniquely defined because of the non-degeneracy of the inner
	product $g$. This is because if we let $\alpha \in \bigwedge\nolimits^{k+2} V^*$ and assume there 
	would be a second operator $\widetilde{\Lambda}$ admitting the same adjunction property, such 
	that $\Lambda(\alpha) \neq \widetilde{\Lambda}(\alpha)$, it would be for all 
	$\beta \in \bigwedge\nolimits^k V^*$
	\begin{align*}
		g(\Lambda(\alpha) - \widetilde{\Lambda}(\alpha), \beta) = g\left(\Lambda (\alpha), \beta\right) -
		g(\widetilde{\Lambda}(\alpha),\beta) = g(\alpha, L(\beta)) - g(\alpha, L(\beta)) = 0.
	\end{align*}
	Thus, the non-degeneracy of $g$ can be used to obtain $\Lambda (\alpha)-\widetilde{\Lambda}(\alpha) = 0$,
	which is a contradiction to our assumption.
\end{rem}
To properly define the next operator now, it is necessary to discuss the existence of a volume
element in $\bigwedge^n V^*$ first.
\begin{rem}
	\label{loc-theory:volume-form-locally}
	We already know that $V$ has a natural orientation, which we are going to call $\sigma$ for now.
	As mentioned in \cite[p.\,83 below Theorem 4-6]{Spivak1965}, there exists a unique volume form 
	$\vol \in \bigwedge^nV^*$, such that $\vol(v_1, \dots, v_n)= 1$ whenever $(v_1, \dots, v_n)$ is 
	an orthonormal basis of $V$ that is positively orientated with respect to $\sigma$. This volume form can 
	be given as $\vol = v^1 \wedge \dots \wedge v^n$.
	
	Now, we want to show that $\vol = \frac{1}{m!} \omega^m$. %with $\omega \in \bigwedge^2 V^*$
	%being the fundamental form associated to $(V,g,I)$.
	Therefore, let $(w_1,\dots,w_m)$ be a complex orthonormal basis of $(V,I)$ with respect to the
	hermitian form $h$. A simple calculation proves that the induced real basis $(w_1,I(w_1),\dots,w_m,I(w_m))$ 
	is orthonormal with respect to the inner product $g$ on $V$. We can calculate for all $w_j$ and $w_k$
	\begin{align*}
		\omega(w_j,I(w_k)) = g(I(w_j),I(w_k)) = g(w_j,w_k) = \delta_{jk}
	\end{align*}
	and also
	\begin{align*}
		\omega(w_j,w_k) = g(I(w_j),w_k) = 0.
	\end{align*}
	Let $I(w_j)^*$ denote the dual basis vector of $I(w_j)$. With the above calculation, we conclude
	that $\omega = \sum_{j=1}^{m} w^j \wedge I(w_j)^*$ and therefore it is
	\begin{align*}
		\omega^m = \Big(\sum_{j=1}^{m} w^j \wedge I(w_j)^*\Big)^m= m! \cdot \big(w^1 \wedge I(w_1)^*\big)
		\wedge \dots \wedge \big(w^m\wedge I(w_m)^*\big) 
		= m! \cdot \vol.
	\end{align*}
\end{rem}
This concludes our collection of the necessary elements to define the Hodge star operator.
\begin{defn}[Hodge star operator]
	The \emph{Hodge star operator} on $\bigwedge^k V^*$ is defined as a linear mapping 
	$\hodgestar: \bigwedge^k V^* \rightarrow \bigwedge^{n-k}V^*$, such that for all 
	$\alpha,\beta \in\bigwedge^kV^*$, it is
	\begin{align}
		\label{loc-theory:eq:hodge-star}
		\alpha \wedge \hodgestar\beta = g(\alpha,\beta) \cdot \vol.
	\end{align}
	The $(n-k)$ form $\hodgestar \beta$ is called the \emph{Hodge dual} of $\beta$.
\end{defn}
Note that \Cref{loc-theory:eq:hodge-star} uniquely defines the Hodge dual because for $r,s \in
\mathbb{N}$ with $r + s = n$, the exterior product defines a non-degenerate pairing 
\begin{align*}
	\bigwedge\nolimits^r V^* \times \bigwedge\nolimits^s V^* \rightarrow \bigwedge\nolimits^{n} V^*.
\end{align*}

To show that such an operator exists, we choose an orthonormal basis $(v_1,\dots,v_n)$ of 
$V$ that is positively oriented with respect to the natural orientation of $V$. With
\Cref{loc-theory:rem:preserving-orthonormality}, we already know that the induced basis $(v^J)_{J}$ of
$\bigwedge^kV^*$ is orthonormal as well. For an arbitrary permutation $\tau \in S_n$, we set the Hodge star
operator to map as follows:
\begin{align}
	\label{loc-theory:eq:explicit-hodge-star}
	\hodgestar\big(v^{\tau(1)}\wedge\dots\wedge v^{\tau(k)}\big) = \sign(\tau) \cdot \;v^{\tau(k+1)}
	\wedge\dots\wedge v^{\tau(n)}.
\end{align}
With this definition, it is
\begin{align*}
	v^{\tau(1)}\wedge\dots\wedge v^{\tau(k)} \wedge \hodgestar\big(v^{\tau(1)}\wedge\dots\wedge v^{\tau(k)}\big) &= \sign(\tau) \cdot v^{\tau(1)} \wedge\dots\wedge v^{\tau(n)}\\
	&= \sign(\tau)^2\cdot v^1\wedge \dots \wedge v^n \\&= 1 \cdot \vol.
\end{align*}
At the same time, it is for all $v^{j_1}\wedge\dots\wedge v^{j_k} \neq \pm v^{\tau(1)}\wedge\dots\wedge v^{\tau(k)}$
\begin{align*}
	v^{j_1}\wedge\dots\wedge v^{j_k} \wedge  \hodgestar\big(v^{\tau(1)}\wedge\dots\wedge
	v^{\tau(k)}\big) = 0.
\end{align*} This is because if we assume without loss of generality that $j_1\neq\dots\neq j_k$,
then there exists at least one $s\in\mathbb{N}$ with $(k+1) \leq s \leq n$, such that $\tau(s) \in \{j_1,\dots,j_k\}$.
This shows that the above mapping indeed explicitly defines the Hodge star. See also \cite[p.\,56f]{Schnell2012}, 
where we have later discovered a very similar calculation.
\begin{prop}[Properties of the Hodge star {\cite[Proposition 1.2.20]{Huybrechts2004}}]\;\\
	\label{loc-theory:lm:property-hodge-star}
	Among others, the Hodge star operator has the following properties:
	\begin{enumerate}
		\item The Hodge star operator on $\bigwedge^kV^*$ is an isometric isomorphism, i.e. it is
		bijective and for all $\alpha, \beta \in \bigwedge^kV^*$, it is $g(\alpha,\beta) = g(\hodgestar\alpha,\hodgestar\beta)$.
		\item For all $\alpha \in \bigwedge^kV^*$, it is $\hodgestar^2\alpha = (-1)^k \alpha$. In particular, it is
		$\hodgestar^{-1} = (-1)^k \hodgestar$.
		\label{loc-theory:lm:property-hodge-star-one}
	\end{enumerate}
\end{prop}
\begin{proof}
	With the explicit definition in \Cref{loc-theory:eq:explicit-hodge-star}, it is quite easy to determine
	that the Hodge star operator maps any orthonormal basis to an orthonormal basis, and this proves
	property (1).
	For the proof of property (2), we use a local version of the calculation in  \cite[Lemma 5.5]{Voisin2002}. 
	We can apply property (1) and calculate for all $\alpha,\beta \in\bigwedge^kV^*$
	\begin{align*}
		\alpha\wedge\hodgestar\beta = g(\alpha, \beta) \cdot \vol = g(\hodgestar\alpha,\hodgestar\beta)
		\cdot \vol = g(\hodgestar\beta,\hodgestar\alpha) \cdot \vol = \hodgestar\beta \wedge \hodgestar^2
		\alpha.
	\end{align*}
	With $n-(n-k) = k$, we know $\hodgestar^2 \alpha$ is a $k$-form again. As $\hodgestar\beta$ is a
	$(n-k)$ form, we obtain
	\begin{align*}
		\hodgestar\beta \wedge \hodgestar^2\alpha &= (-1)^{k(n-k)} \hodgestar^2 \alpha \wedge \hodgestar
		\beta \\&= (-1)^{k(2m-k)} \hodgestar^2\alpha\wedge\hodgestar\beta\\ &= (-1)^k \hodgestar^2 \alpha
		\wedge \hodgestar\beta.
	\end{align*}
	Given that this holds for all $\beta \in \bigwedge^kV^*$, the second property is already proven.
\end{proof}
Additionally, there is an interesting relation between the Hodge star operator and the Lefschetz and
dual Lefschetz operators, which is going to be useful later.
\begin{lm}[{\cite[Lemma 1.2.23]{Huybrechts2004}}]
	\label{loc-theory:lm:formula-for-the-dual-lefschetz-operator}
	For $\alpha \in \bigwedge^{k+2} V^*$ the image under the dual Lefschetz operator $\Lambda$ can be
	explicitly calculated as
	\begin{align*}
		\Lambda (\alpha) = \big(\hodgestar^{-1} \circ \,L \circ \hodgestar\big)(\alpha) = \big( (-1)^k
		\hodgestar \circ \,L \circ \hodgestar \big)(\alpha).
	\end{align*}
\end{lm}
\begin{proof} We expand the proof of the given lemma in \cite{Huybrechts2004}.
	Let $\beta \in \bigwedge^kV^*$. Using the definition of the Hodge star and the definition of the 
	Lefschetz operator, we can calculate
	\begin{align*}
		g(\alpha, L\beta)\cdot \vol = g(L\beta,\alpha) \cdot \vol = L\beta \wedge \hodgestar \alpha =
		\omega \wedge \beta \wedge \hodgestar \alpha.
	\end{align*}
	As $\omega$ is a $2$-form and the wedge product is associative, we have
	\begin{align*}
		g(\alpha, L\beta)\cdot \vol  = (-1)^{2k} \beta \wedge \omega \wedge \hodgestar\alpha = \beta
		\wedge (\omega \wedge \hodgestar\alpha).
	\end{align*}
	Applying the definition of the Lefschetz operator again and using the definition of the Hodge star
	operator yields 
	\begin{align*}
		g(\alpha, L\beta)\cdot \vol  &=\beta \wedge L (\hodgestar \alpha) \\
		&=  \beta \wedge (\hodgestar \hodgestar^{-1}) L (\hodgestar \alpha) \\
		&= \beta \wedge \hodgestar (\hodgestar^{-1} (L (\hodgestar \alpha))) \\
		&= g\big(\beta, (\hodgestar^{-1}( L (\hodgestar \alpha)))\big) \cdot \vol \\
		&=g(\hodgestar^{-1} (L (\hodgestar \alpha)), \beta)\cdot\vol.
	\end{align*}
	Thus, we have shown the equality 
	$g(\Lambda \alpha, \beta) = g(\alpha,L\beta) = g(\hodgestar^{-1}(L(\hodgestar \alpha)), \beta)$. 
	Since this holds for all $k$-forms $\beta$, the non-degeneracy of $g$ proves the first equality 
	of the statement. The second equality then follows directly with
	\creflmpart{loc-theory:lm:property-hodge-star}{loc-theory:lm:property-hodge-star-one}.  % see custom command in main_thesis
\end{proof}
Similar to the $\mathbb{C}$-linear extension of the Lefschetz operator, we will also need the
$\mathbb{C}$-linear extension of the Hodge star operator. 
\begin{defn}
	The $\mathbb{C}$-linear extension of the Hodge star 
	$\hodgestar_\mathbb{C} :\bigwedge^kV_\mathbb{C}^* \rightarrow \bigwedge^{n-k} V_\mathbb{C}^*$ 
	is defined such that for all $\alpha,\beta \in \bigwedge^kV^*_\mathbb{C}$, we have
	\begin{align*}
		\alpha \wedge \hodgestar_\mathbb{C}\overline{\beta} = h(\alpha, \beta) \cdot \vol.
	\end{align*}
	Note that this expression is meaningful because 
	$\vol = \frac{1}{m!} \omega^m \in \bigwedge^nV^*\cap \bigwedge^{m,m}V^*$.
	Since the hermitian form $h$ on $\bigwedge^kV_\mathbb{C}^*$ was previously defined as the
	sesquilinear extension of the inner product $g$ on $\bigwedge^kV^*$, it is immediate that this is indeed 
	the $\mathbb{C}$-linear extension of the real Hodge star operator.
\end{defn}
There is also the $\mathbb{C}$-linear extension 
$\Lambda_\mathbb{C}: \bigwedge^{k+2}V_\mathbb{C}^*\rightarrow \bigwedge^{k} V_\mathbb{C}^*$
of the dual Lefschetz operator. Using the explicit formula in 
\Cref{loc-theory:lm:formula-for-the-dual-lefschetz-operator}, this extension is given as
\begin{align*}
	\Lambda_\mathbb{C} = (-1)^k \hodgestar_\mathbb{C} \circ L_\mathbb{C} \circ \hodgestar_\mathbb{C}.
\end{align*}
We can use the same calculation as in the proof of \Cref{loc-theory:lm:formula-for-the-dual-lefschetz-operator} 
to show that this is indeed the adjoint operator to $L$ with respect to the hermitian form $h$.
\begin{nota}
	It is common practice to abuse the notation, denoting the complex extensions 
	$L_\mathbb{C},\hodgestar_\mathbb{C}$ and $\Lambda_\mathbb{C}$ as $L, \hodgestar$ 
	and $\Lambda$, respectively.
\end{nota}
